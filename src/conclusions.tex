\section{Conclusions}\label{sec:conclusions}

Revisiting the software citation principles in the view of current approaches and technologies in HEP provides a structure for starting community guidelines

\begin{enumerate}
    \item \textbf{Importance}: As a field HEP understands software is important, but improvements could be made on views towards software as a research products.
    \item \textbf{Credit and Attribution}: Improving in HEP, but can leverage software friendly journals (i.e., JOSS) to help accelerate this.
    \item \textbf{Unique Identification}: Use of Zenodo archives is common in HEP, which provides well integrated tooling for DOI generation.
The use of \texttt{CITATION.cff} files in software repositories can help as well.
    \item \textbf{Persistence}: Zenodo provides long term archival of source code and project metadata.
    \item \textbf{Accessibility}: HEP is becoming more FAIR~\cite{FAIR-paper,chue_hong_neil_p_2022_6623556} focused, bringing with it an increased focus on accessibility.
As \texttt{CITATION.cff} provides a common framework for metadata, adopting it as a community standard for software citation information allows for greater accommodation and discovery by citation discovery tooling.
    \item \textbf{Specificity}: Version numbers of software should be included in \texttt{CITATION.cff} files.
\end{enumerate}

it is seen that there are both social and technical tooling challenges to be addressed to reach HEP community guidelines and recommendations for software citation.
While there exist multiple practices towards software citation in the HEP community today, this should not be viewed as a large challenge towards global community standards adoption as variations in homogeneity of practice are common even in journal publication.
The community wide agreement that software citation is important, should be practiced more often, and provides both social and technical benefits provides sufficient motivation to develop HEP community wide recommendations in the near future.
