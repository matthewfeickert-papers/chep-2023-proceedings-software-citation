In November 2022, the HEP Software Foundation (HSF) and the Institute for Research and Innovation for Software in High-Energy Physics (IRIS-HEP) organized a workshop on the topic of Software Citation and Recognition in HEP.
The goal of the workshop was to bring together different types of stakeholders whose roles relate to software citation and the associated credit it provides in order to engage the community in a discussion on: the ways HEP experiments handle citation of software, recognition for software efforts that enable physics results disseminated to the public, and how the scholarly publishing ecosystem supports these activities.
Reports were given from the publication board leadership of the ATLAS, CMS, and LHCb experiments and HEP open source software community organizations (ROOT, Scikit-HEP, MCnet), and perspectives were given from publishers (Elsevier, JOSS) and related tool providers (INSPIRE, Zenodo).
This paper summarizes key findings and recommendations from the workshop as presented at the 26th International Conference on Computing In High Energy and Nuclear Physics (CHEP 2023).
