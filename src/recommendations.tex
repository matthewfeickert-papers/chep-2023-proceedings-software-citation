\section{Recommendations}\label{sec:recommendations}

\subsection{Publishing Community}\label{sec:publishers}

Following the state of software citation in the HEP community, views and recommendations from INSPIRE, Elsevier, and JOSS were shared given their different roles related to scientific publishing and citation.
INSPIRE is an integral part of how HEP interacts with publications, related metadata, and acquires updated citation information as tracked submissions move from preprints through publication and having these capabilities for the citation information for software in HEP would be a technical boon.
While INSPIRE currently only handles software papers there are plans to add support for data products and software in the future, initially by harvesting metadata from relevant trusted repositories (e.g. INSPIRE HEP Zenodo community, HEPData, CERN OpenData).
This information would be gathered by software digital object identifier (DOI), and could be aggregated across multiple releases of the same software, and so it is important that software projects that seek citations in the future provide DOIs now (Principle 3).


% INSPIRE
% Currently handles software papers, but has plans to add support for Data and Software
% Citations would be tracked and counted by DOI
% Elsevier
% Community needs to reach consensus on how to cite software, and share outcome with publishers (won't take lead)
% Publishers can better instruct editors and referees what publishers expect from them
% Journal of Open Source Software
% In addition to incentivizing high quality software, JOSS can help bridge the gap
% "recognize that for most researchers, papers and not software are the currency of academic research"
