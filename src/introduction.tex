\section{Introduction}\label{sec:introduction}
Software is a research product --- an asset created as a byproduct of scientific research --- that is ubiquitous and necessary to physics research, though it is not always given the same levels of importance and scholarly weight as other research products like publications and data products~\cite{Cranmer:2021urp}.
In November 2022, the HEP Software Foundation (HSF) and the Institute for Research and Innovation for Software in High-Energy Physics (IRIS-HEP)~\cite{S2I2HEPSP,IRISHEPWEB} organized a topical workshop on software citation and recognition in the field of high energy physics (HEP)~\cite{software_citation_workshop_report,software_citation_indico}.
The goal of the workshop was to provide a community discussion around ways in which HEP experiments handle citation of software and recognition for software efforts that enable physics results disseminated to the public.
The workshop participants and primary presentations were from the LHC experiments that are primary stakeholders in IRIS-HEP operations: ATLAS, CMS, and LHCb; the particle physics software development communities: ROOT Team, Scikit-HEP~\cite{Rodrigues:2020syo}, MCnet, and IRIS-HEP; as well as the scientific publishing community and ecosystem most involved with HEP: Elsevier, the Journal of Open Source Software (JOSS)~\cite{smith_journal_2018}, and INSPIRE~\cite{INSPIRE}.

The principles of software citation that the HEP community is interested in engaging with are those established by the FORCE11 Software Citation working group~\cite{smith_software_2016}.
These principles are defined as:

\begin{enumerate}
    \item \textbf{Importance}: Software should be considered a legitimate and citable product of research.
Software citations should be accorded the same importance in the scholarly record as citations of other research products, such as publications and data; they should be included in the metadata of the citing work, for example in the reference list of a journal article, and should not be omitted or separated.
Software should be cited on the same basis as any other research product such as a paper or a book, that is, authors should cite the appropriate set of software products just as they cite the appropriate set of papers.
    \item \textbf{Credit and Attribution}: Software citations should facilitate giving scholarly credit and normative, legal attribution to all contributors to the software, recognizing that a single style or mechanism of attribution may not be applicable to all software.
    \item \textbf{Unique Identification}: A software citation should include a method for identification that is machine actionable, globally unique, interoperable, and recognized by at least a community of the corresponding domain experts, and preferably by general public researchers.
    \item \textbf{Persistence}: Unique identifiers and metadata describing the software and its disposition should persist --- even beyond the lifespan of the software they describe.
    \item \textbf{Accessibility}: Software citations should facilitate access to the software itself and to its associated metadata, documentation, data, and other materials necessary for both humans and machines to make informed use of the referenced software.
    \item \textbf{Specificity}: Software citations should facilitate identification of, and access to, the specific version of software that was used.
Software identification should be as specific as necessary, such as using version numbers, revision numbers, or variants such as platforms.
\end{enumerate}

In 2023 the global research community now has these principles, citation policies from journal publishers, modern open source tooling to facilitate the generation of software citations.
There has been growing movement among research software developers, research paper authors, and journal reviewers and editors~\cite{smith_journal_2018} towards an increase in software citation.
For the HEP community it is important to understand the current state (as of 2023) of software citation norms and culture in the field and how its importance can be conveyed and supported through community tooling, standards, and practices.

% Software should be cited on the same basis as any other research product such as a paper or a book, that is, authors should cite the appropriate set of software products just as they cite the appropriate set of papers.
