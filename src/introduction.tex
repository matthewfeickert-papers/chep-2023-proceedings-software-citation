\section{Introduction}\label{sec:introduction}
Software is a research product --- an asset created as a byproduct of scientific research --- that is ubiquitous and necessary to physics research, though it is not always given the same levels of importance and scholarly weight as other research products like publications and data products~\cite{Cranmer:2021urp}.
In November 2022, the HEP Software Foundation (HSF) and the Institute for Research and Innovation for Software in High-Energy Physics (IRIS-HEP)~\cite{S2I2HEPSP,IRISHEPWEB} organized a topical workshop on software citation and recognition in the field of high energy physics (HEP)~\cite{software_citation_workshop_report,software_citation_indico}.
The goal of the workshop was to provide a community discussion around ways in which HEP experiments handle citation of software and recognition for software efforts that enable physics results disseminated to the public.
The workshop participants and primary presentations were from the LHC experiments that are primary stakeholders in IRIS-HEP operations: ATLAS, CMS, and LHCb; the particle physics software development communities: ROOT Team, Scikit-HEP~\cite{Rodrigues:2020syo}, MCnet, and IRIS-HEP; as well as the scientific publishing community and ecosystem most involved with HEP: Elsevier, the Journal of Open Source Software (JOSS)~\cite{smith_journal_2018}, and INSPIRE~\cite{INSPIRE}.

The principles of software citation that the HEP community is interested in engaging with are those established by the FORCE11 Software Citation working group~\cite{smith_software_2016}.
These principles, briefly summarized, are:

\begin{enumerate}
    \item Importance: Software should be viewed as a legitimate research product and be treated as an independent citable asset that carries the same level of importance as other research products such as publications or data.
    \item Credit and Attribution: The act of citation of software should facilitate scholarly credit and attribution at the same level as citation of other research products.
    \item Unique Identification: A software citation should include a method of identification that is machine actionable, globally unique, interoperable, and that is recognized by a community of relevant domain experts, and ideally the general research community.
    \item Persistence: The unique identifies and metadata associated with a software citation should persist beyond the software lifecycle, to provide long term relevant information.
    \item Accessibility: Software citation should facilitate access to the software itself as well as its associated metadata, documentation, associated data products, and other relevant metadata and materials for use of the software.
    \item Specificity: A software citation should facilitate identification of, as well as access to, the specific version of software used in research.
\end{enumerate}

% Software should be cited on the same basis as any other research product such as a paper or a book, that is, authors should cite the appropriate set of software products just as they cite the appropriate set of papers.
