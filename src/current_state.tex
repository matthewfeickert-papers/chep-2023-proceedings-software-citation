\section{Current State of Software Citation in HEP}\label{sec:current_state}

\subsection{LHC Experiments}\label{sec:lhc_experiments}

To understand the current state of software citation in the field reports from the ATLAS, CMS, and LHCb experiments were given that summarized the experiments current standards and practices and future plans.
ATLAS takes the approach of using a ``catch-all'' citation of all ATLAS software and firmware through the citation of an ATLAS public note that ``briefly describes the software and provides links to dynamic and persistent repositories wherein the code resides''.~\cite{ATL-SOFT-PUB-2021-001}
This public note is then cited in many ATLAS papers.
ATLAS does additionally cite the paper on the ATLAS detector simulation software~\cite{SOFT-2010-01} as well as GEANT4~\cite{GEANT4:2002zbu}, and the Monte Carlo simulation generators~\cite{Sjostrand:2007gs,Sjostrand:2014zea,Alwall:2014hca,Sherpa:2019gpd}.
In terms of statistical analysis ATLAS cites the methodology papers that describe the techniques being implemented in software, but in general does not cite the actual software that implements the techniques and is used for physics analysis, with the notable exception of machine learning libraries~\cite{chollet2015keras,tensorflow2015-whitepaper}.
Citation practices are not uniformly consistent in the experiment though, with some physics groups beginning to regularly cite statistical libraries with clear citation guidelines~\cite{pyhf,pyhf_joss}.

CMS similarly has an established culture of regularly and consistently citing the Monte Carlo generators, GEANT4, and machine learning tools.
However, they note that they could improve the citation of the software tha CMS itself produces, both in experimental internal notes and documentation as well as scientific publications.
CMS also expressed positive views towards starting practices of publishing papers --- either as CMS Collaboration publications or as limited authorship papers from the CMS Software and Computing Group --- on CMS software, bringing with it increased visibility of scientific software development, documentation standards, and references of software version information.

LHCb has taken a more proactive stance on software citation following recommendations presented at the CHEP 2018 Conference~\cite{CHEP-2018-recommendations} by providing an internal LHCb software citation starting template of citations for software commonly used in analysis.
Analysis teams are then encouraged to revise the template with the citations of the software used in the analysis with the goal that all high-level software used is properly cited.
These practices are encouraged in the collaboration, but not explicitly required, and so analysis teams may require citation guidelines to be provided.
LHCb also noted that the citation practices of the HEP community are largely due to cultural norms rather than technical challenges, and that while LHCb strives to be citing more software in the future having LHC community recommendations on software citation would be useful for motivating better practices.
